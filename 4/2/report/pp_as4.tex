\documentclass[11pt, oneside]{article}   	% use "amsart" instead of "article" for AMSLaTeX format
\usepackage{geometry}                		% See geometry.pdf to learn the layout options. There are lots.
\geometry{letterpaper}                   		% ... or a4paper or a5paper or ... 
%\geometry{landscape}                		% Activate for for rotated page geometry
%\usepackage[parfill]{parskip}    		% Activate to begin paragraphs with an empty line rather than an indent
\usepackage{graphicx}				% Use pdf, png, jpg, or eps� with pdflatex; use eps in DVI mode
								% TeX will automatically convert eps --> pdf in pdflatex		
\usepackage{amssymb}
\usepackage{amsmath}

\title{Assignment 4}
\author{
Jiannan Guo\\
Ramkumar Rajagopalan\\
Niklas Semmler\\
Christos Tsakiroglou\\
}
%\date{}							% Activate to display a given date or no date

\begin{document}
\maketitle
\section{Question A}
\begin{verbatim}
void multiply_vector_matrix(val, col_ind, row_ptr, vector, res) {
    /*
     * res = Matrix * vector 
     *
         * Matrix in CSR format:
     * val - values
     * col_ind - column indices 
     * row ptr - row pointers
     */
    int lower_bound = row_ptr[0]
    int upper_bound = 0
    int row = 0

    for (i = 1; i < len(row_ptr); i++)
        upper_bound = row_ptr[i]
        for (j = lower_bound; j < upper_bound; j++)
            res[row] += vector[col_ind[j]] * val[j]
        lower_bound = upper_bound
        row++

    return
}
\end{verbatim}

\section{Question B}
One first idea is to distribute the work of the matrix multiplication $A\times b=c$ by row, meaning that each process gets $n$ rows of $A$ and the whole $b$ vector. However, this process may result in highly unbalanced workload, since we don�t take into account the actual number of non-zero elements in each row. Rather than dividing A matrix evenly by rows, we could distribute the rows based on their non-zero elements according to row\_ptr, meaning that we calculate $\frac{length(val)}{p}$, where $p$ is number of processes, and try to distribute a number of non-zero elements as close to that as possible without cutting a row into smaller pieces.
\subsection{Two inconveniences}
\begin{itemize}
\item Workload unbalanced\\
In an extreme case of A:
\begin{equation*}
\begin{bmatrix}
1&1&1&1\\
0&0&0&0\\
0&0&0&0\\
0&0&0&0\\
\end{bmatrix}
\end{equation*}
If first 2 rows are distributed to process 1, and last 2 rows for process 2, then, process 2 will have nothing to calculate at all.
\item Since $N\times N$ matrix $A$ is sparse but not banded matrix, the adjacency matrix of $A$ might have a large bandwidth (in extreme case, $N$). But for a certain row of $A$, there might be plenty of zero gaps inside due to the sparseness. In this case, access to the memory where vector $b$ is stored will be randomized to a large degree, which will reduce the speed.
\end{itemize}
To account for this inequality we evenly share the row\_ptr instead of rows. Below you can find an example for two threads:
\begin{verbatim}
int lower_bound = row_ptr[0]
for (i = 1; i < len(row_ptr)/2; i++)
    upper_bound = row_ptr[i]
    for (j = lower_bound; j < upper_bound; j++)
        res[row] += vector[col_ind[j]] * val[j]
    lower_bound = upper_bound
    row++

lower_bound = row_ptr[len(row_ptr)/2+1]
for (i = len(row_ptr)/2+2; i < len(row_ptr); i++)
    upper_bound = row_ptr[i]
    for (j = lower_bound; j < upper_bound; j++)
        res[row] += vector[col_ind[j]] * val[j]
    lower_bound = upper_bound
    row++
\end{verbatim}

\section{Question C}
\subsection{Implications of the band structure on the multiplication and describe an improved version!}
Here we have a band-structured matrix, trying to minimize the maximum index difference between non-zero elements, thus minimizing the bandwidth of the matrix. The smaller the bandwidth is, the less the complexity and the faster the calculation becomes.\\
Assuming an $N\times N$ band matrix with a bandwidth of $B$, we store it in memory as an $N\times B$ matrix. Thus, $N\times (N-B)$ amount of memory is already saved. Memory accesses are reduced and number of operations is also reduced.\\
As far as the parallelization is concerned, the non-zero part of the band matrix is divided into smaller sub-matrices, which are then distributed to different processes. However, there are always elements outside the diagonals which need to be communicated between processes. Moreover, each process only needs to know a part of vector b, which actually consists of values stored successively, so only needs to fetch a block of memory. On the contrary, at the simple sparse matrix case, vector b had to be accessed in various memory addresses, increasing the memory blocks accessed. A more detailed explanation follows.\\

\subsection{Analyze and outline...}
Analyze and outline the effects on the Jacobi implementation described above when using this improved version.\\
Take the equation below as an example (with band structure matrix $bw=4$):
\begin{equation}
\begin{bmatrix}
a_{11} & a_{12} & a_{13} & a_{14} \\
a_{21} & a_{22} & a_{23} & a_{24} \\
a_{31} & a_{32} & a_{33} & a_{34} & c_{31} \\
a_{41} & a_{42} & a_{43} & a_{44} & 0 & c_{42}\\
&& d_{13} & 0 &b_{11} & b_{12} & b_{13} & b_{14} \\
&&& d_{24} &b_{21} & b_{22} & b_{23} & b_{24} \\
&&&&b_{31} & b_{32} & b_{33} & b_{34} \\
&&&&b_{41} & b_{42} & b_{43} & b_{44} \\
\end{bmatrix}
\begin{bmatrix}
x_{1}\\
x_{2}\\
x_{3}\\
x_{4}\\
y_{5}\\
y_{6}\\
y_{7}\\
y_{8}\\
\end{bmatrix}
=
\begin{bmatrix}
n_{1}\\
n_{2}\\
n_{3}\\
n_{4}\\
m_{5}\\
m_{6}\\
m_{7}\\
m_{8}\\
\end{bmatrix}
\end{equation}
It can be rewritten as:
\begin{equation}
\begin{bmatrix}
A & C\\
D & B\\
\end{bmatrix}
\begin{bmatrix}
X\\
Y\\
\end{bmatrix}
=
\begin{bmatrix}
N\\
M\\
\end{bmatrix}
\end{equation}
With this structure, assuming we parallelize Jacobi iterations with 2 processes, with first one holding $A\times X=N$ and second one holding $B\times Y=M$. But there are still sparse matrices $C$ and $D$ which are interface elements, where the communications occur. But domain decomposition storage, communication is reduced in a way that every block of vector will only need to exchange message with adjacent ones. In this case, $m_{5}$ and $m_{6}$ will be sent to second process for next iterative step and vice versa.
\end{document}  